\documentclass{article}
\input{preamble}

\title{Homework 1}
\author{Kairos Scoleri}

\begin{document}
\maketitle

\begin{enumerate}

%%%% problem 1
\item Genetically, all cats are either black or orange. The trait is sex-linked (on the X chromosome) and co-dominant: male cats (XY) can either be black (BY) or orange (OY) only; female cats (XX) can be black (BB) orange (OO) or calico (BO or OB). The prevalence of the orange gene in the population is about 20\%.
\begin{enumerate}[(a)]
    \item What percent of male cats are orange? Female cats?
    \item If an orange tom and a calico kitty have kittens, what percentage of sons are orange? What percentage of daughters are orange?
    \item If they have 8 kittens, what is the probability that they will all be orange? What is the probability that exactly 4 will be orange?
\end{enumerate}

%%%% problem 2
\item The elasticity of a rubber band can be modeled as a polymer made up of $N$ subunits each of length $x$. Each link is equally likely to point right or left (see \figref{fig:rubberband}). One end is fixed at the origin.


\begin{figure}[h]
\centering
\includegraphics[width=\linewidth]{Statistical Mechanics/images/hw 1 rubberband.png}
\caption{} 
\label{fig:rubberband}
\end{figure}
\begin{enumerate}[(a)]
    \item How many different arrangements yield a length $L=2 mx$, where $m$ is an integer?
    \item Give an expression for the entropy assuming large $N$.
    \item Get an expression for the retractive force. Hint: use $\dd U=T\: \dd S+f \: \dd L$, $\dd F=-S\:\dd T+f \: \dd L$, then use the approximate version of $S$ calculated in part (b).
\end{enumerate}

%%%% problem 3
\item Evaluate $i!$

%%%% problem 4
\item An inveterate liar throws a die and tells you it came up 6. If he lies 45\% of the time, what is the chance that the real result was actually 6?

%%%% problem 5
\item The geometric series is of great utility in statistical mechanics:
\[\sum_{n=0}^\infty x^n =\frac{1}{1-x}\]
\begin{enumerate}[a.]
    \item Start with this result to sum the series
    \[\sum_{n=0}^\infty nx^n\]
    \item Start with this result to sum the series
    \[\sum_{n=0}^\infty n^2x^n\]
\end{enumerate}

%%%% problem 6
\item Consider 2 systems $\mathbf{A}$ and $\mathbf{B}$, each composed of 2 distinguishable particles. The total energy is fixed as $E=E_A +E_B =5$ (arbitrary units) where each particle can exist in a state of integer energy (0 to 5). Particle exchange does not take place. Calculate the entropy in the cases where 
\begin{enumerate}
    \item $E_A =3$ and $E_B =2$
    \item $E_A$ and $E_B$ can be anything as long as $E=5$.
\end{enumerate}



\end{enumerate}
\end{document}
